\chapter*{Resumo}

O uso de dispositivos móveis faz cada vez mais parte do dia-a-dia dos seres humanos. O telemóvel, o leitor de .mp3, o \textit{tablet}, o GPS, a câmara fotográfica, etc. passaram a ser objetos comuns nos bolsos e carteiras. E é precisamente devido às capacidades dos \textit{smartphones}, os telemóveis inteligentes que vieram revolucionar a forma como o telemóvel é visto, deixando este de ser apenas um utensílio para a realização de chamadas e envio de mensagens de texto passando para um sistema completo de agenda, telefone, reprodutor multimédia, câmara fotográfica, navegador GPS, entre muitas outras funcionalidades, que surge a pergunta: Se o telemóvel suporta todas estas funcionalidades, porque não utilizá-lo também para uma das tarefas mais rotineiras de muitas pessoas, os transportes públicos?
\\Apesar de, em Portugal, ser já comum utilizar-se sistemas informatizados e relativamente sofisticados no que toca aos sistemas de bilhética nos transportes públicos, estes estão ainda “presos” a cartões, no Grande Porto o cartão Andante, na Grande Lisboa o cartão Viva, nos comboios o cartão CP. Será a informação armazenada nos cartões demasiado complexa que não possa ser armazenada noutro meio, nomeadamente o telemóvel? A resposta é óbvia, Não! A informação armazenada limita-se ao número de títulos de viagem disponíveis naquele cartão e em caso de um título estar ativo, qual o seu período de validade e a estação de entrada. Toda esta informação pode ser facilmente armazenada no telemóvel, sendo apenas necessário implementar as infraestruturas tecnológicas que deem suporte a este novo modelo.
\\O objetivo desta dissertação é precisamente estudar e conceber a melhor maneira de tirar partido dos telemóveis para substituir os cartões no que toca à bilhética nos transportes públicos da Área Metropolitana do Porto. Isto permitirá não só aumentar a comodidade do passageiro, como proporcionar-lhe um acesso ilimitado e independente de local ou hora a todos os serviços necessários na gestão da sua mobilidade. Estes serviços incluem a compra de títulos de viagem e sua posterior validação, a consulta do saldo da carteira virtual e o saldo de viagens, a visualização do histórico de operações, a visualização do estado atual da viagem validada (tempo restante e estação de entrada, número de zonas, etc.), entre outros.
\\Mas não é só o passageiro que fica a ganhar. Os operadores de transportes públicos de passageiros, para além de acrescentarem valor à sua oferta de serviços, reduzem custos de operação e manutenção, recolhem informação estatística e hábitos de utilização dos passageiros, e mostram vontade de estar na vanguarda no que toca à inovação tecnológica. Existem já vários projetos piloto em execução em vários países, testando tecnologias diferentes, em busca de a melhor solução. É importante referir que cada rede de transportes públicos é única e com características especiais, pelo que não será nunca possível encontrar uma solução que sirva todos os modelos utilizados, sendo necessário desenhar uma solução adequada às necessidades de cada rede.

\chapter*{Abstract}

The use of mobile devices is more and more a daily routine in human beings. Mobile phones, .mp3 players, tablets, GPS, cameras, etc. are now ordinary objects in pockets or bags. And it's precisely due to the capacities of smartphones, the smart mobile phones that revolutionized the way mobile phones are seen, no longer being just a tool to make calls and send text messages, and becoming a complete system of diary, phone, multimedia player, camera, GPS navigator, among others, that the question “If smartphones support all these functionalities, why not use it on one daily task for most people, public transportation?” arises.
\\Although, in Portugal, the use of computerized and relatively sophisticated systems in public transportation ticketing is now a common practice, it is still “stuck” to cards. In Oporto, there is the Andante card, in Lisbon the Viva card and in trains, the CP card. Is the stored information so complex that it couldn’t be stored anywhere else, for instance a smartphone? The answer is obvious, No! The stored information is only the number of remaining traveling titles in that card, and in case one title is active, the expiration time and departure station. All this information can be easily stored in a smartphone, with only the need to develop the technological infrastructures to support this model.
\\The aim of this dissertation is precisely to study and conceive the best way to take advantage of smartphones to replace cards in the Oporto Metropolitan Area public transportation system. This would not only increase passenger’s commodity but also provide him an unlimited and place and time independent access to all the needed services related to his mobility management. These services include purchasing traveling titles, and validating them, checking the virtual wallet balance and the titles balance, visualizing the operations history, visualizing the status of an active title (remaining time, departure station, number of zones, etc.), among others.
\\But the passenger is not the only one who benefits. The passengers public transportation operators, besides adding value to their services offer, reduce operation and maintenance costs, and gather statistical information and use habits from their customers. Moreover, they show the will to be in the edge of technological innovation. There are already some pilot projects running in different countries, searching for the best solution. It is important to have in mind that each public transportation network is unique and with special characteristics, so it will never be possible to find a general solution that fits every model, enforcing the necessity to develop an appropriate solution for each network.
