\chapter{Pagamentos Móveis para Transportes Públicos no Porto}\label{chap:chap3}

\section*{}

O objetivo deste projeto é remover a necessidade de um elemento físico (cartão) nos pagamentos e validações em transportes públicos na Área Metropolitana do Porto, com a implementação de um sistema de pagamento e validação remoto via Internet, utilizando dispositivos móveis com o sistema operativo Android. Para além disso pretende-se recolher informação relativa às infraestruturas mais utilizadas neste âmbito, estudar as principais funcionalidades e operações existentes para implementação na aplicação e desenvolver soluções apropriadas para os problemas existentes.

\section{Introdução}

O modelo de utilização baseia-se no sistema atual, com algumas alterações:
\begin{itemize}
\item Carregamento de saldo feito através de pré-carregamento de uma carteira, através de diversas modalidades de pagamento (MB Phone, PayPal, cartão de crédito/débito, etc.);
\item Compra de títulos de viagem com o saldo disponível na carteira virtual, e posterior ativação dos títulos aquando da entrada no veículo (localização via GPS ou ativação manual por parte do passageiro). Sempre que houver transbordo, o passageiro deve validar novamente o título;
\item Revisão feita pelo agente autorizado tendo em conta o ecrã de verificação gerado no dispositivo do utilizador, contendo informação pertinente tal como o histórico de validações, contendo data/hora, paragem de entrada e linha de entrada, a descrição do título em uso, número sequencial e símbolos de controlo.
\end{itemize}


Este modelo de pagamento e utilização requer um Sistema de Informação complexo e com necessidade de sincronismo e registo de operações em tempo real, sem ocorrência de falhas. É necessário criar também um compromisso entre segurança e conveniência/usabilidade.
\\Atualmente são vários os métodos de pagamento disponíveis no mercado, desde o bilhete tradicional em papel ao sistema Andante, baseado em RFID. No entanto, há ainda uma oportunidade por explorar. Tendo em conta que o telemóvel é cada vez mais um objeto indispensável no dia-a-dia dos seres humanos, tirar partido das suas funcionalidades para viajar em transportes públicos é uma solução bastante conveniente para os utilizadores.
\\Como principais requisitos funcionais, este sistema dispõe de funcionalidades de carregamento da carteira virtual, compra e validação (entrada e saída (opcional)) de títulos de viagem, verificação da validade por parte do revisor; sendo estas as necessidades de um sistema tradicional de transportes públicos.\cite{Buttyan2009}
\\Pretende-se que o projeto permita, numa fase posterior, uma integração com a aplicação MOVE-ME já existente no mercado, servindo como uma implementação de funcionalidades extra da mesma e também que seja possível por parte do administrador, a recolha, processamento e análise de informações relativas às viagens dos utilizadores.

\section{Arquitetura}

O sistema é composto por três componentes fundamentais. A componente servidor (\emph{Server}) que pode ser considerado o centro do sistema, uma vez que é este que disponibiliza os vários serviços, com a qual as outras componentes interagem remotamente. A componente cliente (\emph{Client}), a qual permite ao passageiro interagir diretamente com os serviços disponibilizados. Finalmente, a componente revisor (\emph{Conductor}), a qual permite aos revisores fiscalizar os passageiros que usam este sistema. De referir que, nesta fase, estas duas últimas componentes estão integradas na mesma aplicação, no telemóvel do passageiro, removendo a necessidade de os revisores andarem equipados com dispositivos apropriados.
\\Esta arquitetura segue a típica arquitetura cliente/servidor.

\subsection{Cliente (Client)}

Esta componente é a interface que permitirá aos utilizadores desta plataforma interagir com a mesma. Essa interação será conseguida através do uso de um telemóvel com sistema operativo Android para aceder à aplicação. As principais capacidades desta componente são:
\begin{itemize}
\item Compra de títulos por zonas, permitindo a compra de mais do que um título;
\item Capacidade de armazenamento de títulos de viagem para uso posterior, sendo que os títulos podem ser de diferentes tipologias (diferente número de zonas, diferente modalidade);
\item Validação um título de viagem, escolhendo o título desejado para utilização, sendo feita a comunicação com o servidor e recebendo a confirmação do mesmo;
\item Visualização do saldo da carteira virtual e dos títulos adquiridos;
\item Visualização do histórico de operações;
\item Visualização do histórico de validações de títulos.
\end{itemize}

\subsection{Revisor (Conductor)}

Esta componente é a que permite aos revisores visualizarem a validade dos títulos de viagem em utilização pelos passageiros. Consiste num módulo da aplicação no telemóvel do passageiro que exibe informação detalhada relativa ao título de viagem ativo.

\subsection{Servidor (Server)}

Esta componente irá disponibilizar serviços para os outros componentes da arquitetura. O funcionamento correto desta componente será assegurado por três subcomponentes:
\begin{itemize}
\item Base de dados – Toda a informação que seja necessária guardar será guardada numa base de dados, acessível apenas localmente;
\item Página \web – Esta subcomponente proporcionará um painel de controlo para que o responsável pela administração da plataforma possa gerir todos os aspetos do sistema e ao mesmo tempo será também uma interface que lhe permitirá seguir o progresso dos passageiros;
\item Serviço \web – A maioria da lógica do sistema será processada por esta componente e funcionará como intermediário entre a componente cliente/revisor e a base de dados central.
\end{itemize}

\section{Tecnologias}

\subsection{Android}

A escolha sobre o sistema operativo recaiu sobre o Android por ser a plataforma móvel mais popular no mundo. Com um dispositivo Android, os utilizadores podem usar todos os serviços Google a que estão habituados, para além de mais de 600 mil aplicações e jogos disponíveis na loja virtual Google Play, sendo que muitas das aplicações são gratuitas. Para além disso é possível obter milhões de músicas e livros e também milhares de filmes. Os dispositivos Android são melhorados constantemente com lançamentos de atualizações e novas funcionalidades com bastante frequência. Proporcionam também aos utilizadores uma experiência única e personalização de conteúdos.
\\Uma mais valia é o facto de a aplicação MOVE-ME se encontrar também desenvolvida para este sistema operativo, sendo assim mais fácil a integração.
\\A versão base escolhida será 2.2 (Froyo) pois mais de 98\% dos dispositivos possuem esta versão ou superior e esta versão oferece as funcionalidades necessárias. Ver Tabela~\ref{tab:android}.

\begin{table}[t]
  \centering
  \caption{Distribuição de dispositivos ativos por versão do sistema operativo Android (1 de maio de 2013)\cite{dashboards}}
\begin{tabular}{p{20mm} p{45mm} p{10mm} p{20mm} }
	\hline
\textbf{Versão} & \textbf{Nome de Código} & \textbf{API} & \textbf{Distribuição}\\
	\hline
	\hline
	1.6 & Donut & 4 & 0.1\%\\\hline
	2.1 & Eclair & 7 & 1.7\%\\\hline
	2.2 & Froyo & 8 & 3.7\%\\\hline
	2.3-2.3.2 & \multirow{2}{*}{Gingerbread} & 9 & 0.1\%\\
	2.3.3-2.3.7 & & 10 & 38.4\%\\\hline
	3.2 & Honeycomb & 13 & 0.1\%\\\hline
	4.0.3-4.0.4 & Ice Cream Sandwich & 15 & 27.5\%\\\hline
	4.1.x & \multirow{2}{*}{Jelly Bean} & 16 & 26.1\%\\
	4.2.x & & 17 & 2.3\%\\\hline
\end{tabular}
  \label{tab:android}
\end{table}

\subsection{SQLite}

Para a base de dados de suporte, a utilizar pela aplicação no telemóvel do passageiro, foi escolhida a tecnologia SQLite que é uma biblioteca que implementa um motor de base de dados SQL auto-contido, sem servidor e sem configuração. O código da biblioteca é público e, portanto, gratuito para qualquer uso, comercial ou privado. SQLite é atualmente usado em milhares de aplicações, incluindo projetos de elevada complexidade.
\\O que difere o motor SQLite de outras implementações SQL é o facto de não necessitar de um processo externo de servidor. A leitura e escrita é feita diretamente para ficheiros no disco. Uma base de dados SQL completa, com múltiplas tabelas, índices, gatilhos, vistas, etc. é contida num único ficheiro local. O formato da base de dados é multi-plataforma, permitindo transmissão de ficheiros entre sistemas 32-bit e 64-bit, ou entre sistemas com outras arquiteturas.
\\SQLite é uma biblioteca compacta, mesmo com todas as funcionalidades ativas, a biblioteca ocupa menos de 350kB. Para além disso pode ser corrida em ambientes com pouca memória, tornando-a a escolha ideal para dispositivos móveis. Obviamente, existe um balanço entre utilização de memória e velocidade, embora a performance seja normalmente bastante boa mesmo em ambientes com pouca memória.\cite{sqlite}

\subsection{Microsoft .NET Framework 3.5}

O .NET Framework é o modelo de programação abrangente e consistente da Microsoft para criar aplicações que têm experiências de utilizador agradáveis, comunicações totalmente integradas e seguras e a capacidade de modelar um intervalo de processos de negócio.
\\Inclui uma larga biblioteca e proporciona interoperabilidade de linguagens entre várias linguagens de programação. Os programas desenvolvidos para a .NET Framework são executados num ambiente de software, conhecido como \textit{Common Language Runtime} (CLR), uma máquina virtual de aplicações que oferece serviços tais como segurança, gestão de memória, tratamento de exceções.
\\A biblioteca base oferece interface com utilizador, acesso a dados, ligação a bases de dados, criptografia, desenvolvimento de aplicações \web, algoritmos numéricos, comunicação em rede, etc.

\subsection{Oracle Database XE}

A base de dados do servidor assentará sobre a tecnologia Oracle XE (eXpress Edition) por ser bastante poderosa (apesar de algumas limitações) e muito utilizada a nível mundial, dispondo de licença gratuita e gozar de uma comunidade dedicada, bem como de documentação altamente informativa, através das quais é relativamente fácil aprender e obter soluções para eventuais problemas que possam surgir.

\section{Resumo e Conclusões}

Com este projeto, pretende-se proporcionar aos operadores de transportes públicos da Área Metropolitana do Porto um novo serviço que lhes trará valor e fará aumentar a satisfação dos seus utilizadores. Para além disso, permitir-lhes-á aceder a informação estatística sobre os padrões de utilização dos passageiros e assim fazer possíveis alterações de modo a melhorar o funcionamento da rede.
\\Por outro lado, irá permitir reduzir custos de manutenção, nomeadamente na emissão de cartões, na manutenção das máquinas de venda automática, que passam a ter uma afluência menor e consequentemente um menor desgaste; e sem quaisquer custos de implementação visto que não há necessidade de infraestruturas adicionais às já existentes.
\\Para o cliente, será uma melhoria a nível de comodidade e facilidade de execução das operações que, em muitos casos, são realizadas diariamente. Para além disso, poderá fazê-lo em qualquer altura e em qualquer lugar, não necessitando de se deslocar às máquinas de venda automática ou aos pontos de atendimento. 
