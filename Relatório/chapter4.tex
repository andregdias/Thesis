\chapter{Implementação}\label{chap:implement}

\section*{}

Neste capítulo é especificado de forma detalhada o desenvolvimento e implementação do protótipo desenvolvido, bem como as suas principais características. Pretende-se apresentar as várias fase de desenvolvimento e, em cada uma delas, o trabalho realizado e as conclusões retiradas, tendo em vista uma melhoria constante do produto final.

\section{Especificação de Requisitos}

Para desenvolver uma aplicação que se adequasse às necessidades reais dos utilizadores e das entidades envolvidas, foi necessário analisar o sistema de bilhética atual e perceber quais as ações fundamentais para uma implementação eficaz. Para além disso, foi também necessário avaliar quais as principais funcionalidades que trariam valor ao serem implementadas e que tirariam o máximo proveito das capacidades dos dispositivos móveis. Em anexo encontra-se a especificação detalhada dos requisitos funcionais. Ver Anexo~\ref{rer}.

\subsection{Requisitos Funcionais}

Começando pelos requisitos relacionados com a utilização de autenticação de utilizadores, salvaguardando assim os dados pessoais e permitindo a segurança das operações efetuadas, definiram-se como requisitos funcionais os seguintes:
\begin{itemize}
\item O sistema deve permitir o registo de um novo utilizador;
\item O sistema deve permitir a autenticação de um utilizador já registado;
\item O sistema deve permitir a um utilizador autenticado alterar os seus dados pessoais;
\item O sistema deve permitir a um utilizador autenticado terminar a sessão ativa.
\end{itemize}

Os requisitos funcionais relacionados com o funcionamento de um sistema de bilhética são os seguintes:
\begin{itemize}
\item O sistema deve permitir a compra de títulos por utilizadores autenticados;
\item O sistema deve utilizar os fornecedores de localização do dispositivo móvel para identificar a paragem onde o utilizador autenticado se encontra;
\item O sistema deve permitir ao utilizador autenticado a escolha manual da paragem de entrada;
\item O sistema deve listar as linhas e respetivos sentidos que passam na paragem selecionada pelo utilizador autenticado;
\item O sistema deve permitir ao utilizador autenticado escolher o título a validar, apresentando todos os títulos disponíveis que se adequem à paragem e linha selecionadas;
\item O sistema deve permitir ao utilizador autenticado efetuar a validação do título escolhido, apresentando a paragem limite até à qual pode viajar;
\item O sistema deve permitir a mudança de linha (transbordo), quando existir um título válido;
\item O sistema deve permitir a confirmação da validade do título em utilização, por parte do revisor.
\end{itemize}

Por fim, os requisitos funcionais relacionados com a visualização de informação por parte do utilizador:
\begin{itemize}
\item O sistema deve permitir a consulta do estado atual do título validado;
\item O sistema deve permitir o acesso ao histórico de operações efetuadas pelo utilizador autenticado;
\item O sistema deve permitir o acesso ao histórico de validações realizadas pelo utilizador autenticado;
\item O sistema deve permitir a consulta do saldo de títulos disponíveis;
\item O sistema deve permitir a consulta de saldo da carteira virtual.
\end{itemize}

\subsection{Requisitos Não Funcionais}

Para além dos requisitos acima especificados, foram também delineadas algumas características que a aplicação deve conter:
\begin{itemize}
\item Comunicação - É necessário uma ligação à rede com um acesso estável e com uma largura de banda mínima que permita a plena utilização de todas as funcionalidades disponíveis;
\item Eficiência - Uma normal utilização do sistema requer inúmeros acessos simultâneos à aplicação e, consequentemente, à sua base de dados. Logo, é fulcral que a aplicação esteja estruturada para que a informação seja acedida e apresentada em tempos de resposta mínimos, de modo a que o utilizador veja satisfeitos os seus propósitos de manipulação de informação;
\item Fiabilidade - O sistema deverá garantir a integridade dos dados submetidos;
\item Manutenção - O sistema deverá permitir uma fácil manutenção e adição de novas funcionalidades;
\item Segurança - O sistema deve estar devidamente protegido para que não haja a possibilidade de acessos indevidos a dados confidenciais dos utilizadores.
\item Usabilidade - Pretende-se que a interface da aplicação seja intuitiva e de fácil utilização, para que o utilizador não perca muito tempo no processo de aprendizagem do manuseamento da mesma. É importante também que o número de cliques para execução das ações seja o menor possível, reduzindo assim o tempo de execução. O sistema deverá, também, oferecer mensagens de erro claras e ajuda contextual;
\item Compatibilidade - O sistema deverá funcionar perfeitamente em dispositivos Android com verão 2.2 (API 8) ou superior. Para além disso, o sistema deverá ter uma integração coerente com a aplicação MOVE-ME.

\end{itemize}

\section{Mais uma Secção}


%\begin{lstlisting}[float,language=Java, label=src:mapreduce, caption=Example map and reduce functions for word counting]
%map(String key, String value): 
%// key: document name 
%// value: document contents 
%for each word w in value:
%EmitIntermediate(w, "1");
%
%reduce(String key, Iterator values):
%// key: a word 
%// values: a list of counts 
%int result = 0;
%for each v in values: 
%result += ParseInt(v);
%
%Emit(AsString(result))
%\end{lstlisting}


\section{Resumo ou Conclusões}