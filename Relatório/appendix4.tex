\chapter{Inquérito Inicial} \label{inquerito}

~\\\textbf{Há quanto tempo possui smartphone?}
\begin{itemize}
\item Menos de 6 meses
\item Entre 6 meses e 2 anos
\item Há mais de 2 anos
\end{itemize}

~\\\textbf{Selecione (até 5 tipos) as aplicações que utiliza com mais frequência no smartphone.}
\begin{easylist}[checklist]
&& Livros
&& Negócios/Finanças
&& Educação
&& Transportes
&& Jogos
&& Saúde e Fitness
&& Lifestyle
&& Música
&& Navegação
&& Jornais e revistas
&& Fotos e vídeo
&& Redes sociais
&& Desporto
&& Utilidades
&& Meteorologia
\end{easylist}

~\\\textbf{Em média, com que frequência utiliza transportes públicos?}
\begin{itemize}
\item 1 a 5 vezes/semana
\item 1 a 5 vezes/mês
\item 1 a 10 vezes/ano
\item Raramente
\end{itemize}

~\\\textbf{Que tipo de viagens faz com mais frequência utilizando os transportes públicos?}
\begin{itemize}
\item Urbana
\item Interurbana
\end{itemize}

~\\\textbf{Nas suas viagens urbanas/interurbanas, que meios de transporte utiliza com mais frequência (Se normalmente a sua viagem inclui transbordos entre diferentes meios de transporte, pode selecionar mais que uma opção)?}
\begin{easylist}[checklist]
&& Autocarro
&& Metro
&& Comboio
\end{easylist}

~\\\textbf{Onde compra os bilhetes de viagem para os transportes públicos (pode selecionar mais que uma opção)?}
\begin{easylist}[checklist]
&& Máquinas de Venda Automática
&& Lojas Andante
&& Agentes Payshop e CTT
&& Rede Multibanco
&& Terminais de Operadores Rodoviários
&& Dentro do Autocarro
&& Postos de Atendimento STCP (Hosp. S. João e Bom Sucesso)
&& Bilheteiras CP
&& Internet
&& Outra:  \hspace{0.2cm} \makebox[1.5in]{\hrulefill}
\end{easylist}

~\\\textbf{Qual o meio de pagamento que habitualmente utiliza para pagar os bilhetes?}
\begin{itemize}
\item Notas e moedas
\item Cartão multibanco
\end{itemize}

~\\\textbf{Que tipo de bilhetes costuma comprar com mais frequência?}
\begin{itemize}
\item Títulos Ocasionais
\item Andante 24
\item Assinatura Mensal
\item Outra:  \hspace{0.2cm} \makebox[1.5in]{\hrulefill}
\end{itemize}

~\\\textbf{Que tipo de serviços adicionais relacionados com os transportes públicos costuma utilizar (pode selecionar mais que uma opção)?}
\begin{easylist}[checklist]
&& Consulta de mapas (planos de viagem, linhas, …)
&& Consulta de horários
&& Consulta de tarifários
&& Reclamações
&& Linha de apoio ao cliente
&& Descontos em serviços parceiros do seu operador de transportes
&& SMS Bus
&& Aplicações móveis
&& Outra:  \hspace{0.2cm} \makebox[1.5in]{\hrulefill}
\end{easylist}

~\\\textbf{Que tipos de canais costuma utilizar para aceder aos serviços do seu Operador de Transportes Públicos (informações, mapas, …) (pode selecionar mais que uma opção)?}
\begin{easylist}[checklist]
&& Lojas físicas
&& Paragens/Estações
&& Website
&& Facebook
&& Telefone
&& Email
\end{easylist}

~\\\textbf{Classifique (de 1 a 5) as seguintes afirmações relacionadas com a compra e validação de bilhetes através dos métodos disponíveis actualmente.} (1 – Discordo Totalmente; 2 – Discordo; 3 – Não concordo, nem discordo; 4 – Concordo; 5 – Concordo Totalmente)
\begin{enumerate}
\item Já perdi o meu cartão Andante várias vezes.
\item Tenho dificuldade em perceber que tipo de bilhete (Z2, Z3, Z4..) tenho de comprar para efetuar determinada viagem.
\item Considero fácil comprar bilhetes nas máquinas de venda automática.
\item Já passei por situações em que não tinha dinheiro trocado (notas e moedas) para comprar bilhetes nas máquinas de venda automática.	
\item Não gosto (ou não gostaria) de me deslocar a uma loja física para renovar o meu passe mensal.	
\item Considero fácil validar bilhetes nos validadores.	
\item Gostava de saber quanto tempo me resta de viagem depois de validar o bilhete.	
\item Gostava que o sistema me informasse sobre a paragem até à qual posso viajar com o bilhete que validei.	
\item Gostava de poder guardar mais que um tipo de título diferente (ex: Z2 e Z3) no meu cartão andante.	
\item Gostava de ter acesso ao meu histórico de viagens.	
\item Gostava de saber quanto gasto por mês em transportes públicos.	
\item É mais provável que me esqueça do cartão andante em casa do que do telemóvel.
\end{enumerate}


~\\\textbf{Classifique (de 1 a 5) as seguintes afirmações relacionadas com a compra de bilhetes através do telemóvel.} (1 – Discordo Totalmente; 2 – Discordo; 3 – Não concordo, nem discordo; 4 – Concordo; 5 – Concordo Totalmente) 
\\A compra de bilhetes através do telemóvel significa que em vez de pagar o bilhete andante com notas e moedas, pagaria com o seu telemóvel (através por exemplo de um saldo que tinha disponível para o efeito). Depois do pagamento receberia o cartão andante com as viagens carregadas.
\begin{enumerate}
\item Comprar bilhetes de transporte com o telemóvel é uma boa ideia.	
\item Pretendo utilizar o telemóvel para comprar bilhetes de transporte quando este serviço estiver disponível.	
\item Os pagamentos com o telemóvel são um método de pagamento útil.	
\item Comprar bilhetes de transporte com o telemóvel é compatível com as outras utilizações que faço do telemóvel.	
\item Comprar bilhetes com o telemóvel é compatível com o meu estilo de vida e hábitos.	
\item Sentir-me-ia seguro em comprar bilhetes de transporte com o meu telemóvel.	
\item Os telemóveis são confiáveis o suficiente para a compra bilhetes.	
\item As redes móveis são confiáveis o suficiente para a compra de bilhetes.	
\item Na compra de bilhetes, o risco de ficar sem bateria ou ficar sem rede é baixo.	
\item O risco de haver problemas com a aplicação de compra de bilhetes é baixo.	
\item O risco de eu cometer erros na compra de bilhetes com o telemóvel é baixo.	
\end{enumerate}

~\\\textbf{Classifique (de 1 a 5) as seguintes afirmações relacionadas com a validação de bilhetes através do telemóvel.} (1 – Discordo Totalmente; 2 – Discordo; 3 – Não concordo, nem discordo; 4 – Concordo; 5 – Concordo Totalmente) 
\\A validação de bilhetes através do telemóvel significa que após a compra os bilhetes ficariam guardados no telemóvel e não no cartão andante. A validação seria efetuada através do telemóvel.
\begin{enumerate}
\item Validar bilhetes de transporte com o telemóvel é uma boa ideia.	
\item Pretendo utilizar o telemóvel para validar bilhetes de transporte quando este serviço estiver disponível.	
\item As validações com o telemóvel são um método de validação útil.	
\item Validar bilhetes de transporte com o telemóvel é compatível com as outras utilizações que faço do telemóvel.	
\item Validar bilhetes com o telemóvel é compatível com o meu estilo de vida e hábitos.	
\item Sentir-me-ia seguro em validar bilhetes de transporte com o meu telemóvel.	
\item Os telemóveis são confiáveis o suficiente para a validação de bilhetes.	
\item As redes móveis são confiáveis o suficiente para a validação de bilhetes.	
\item Na validação de bilhetes, o risco de ficar sem bateria ou ficar sem rede é baixo.	
\item O risco de haver problemas com a aplicação de validação de bilhetes é baixo.	
\item O risco de eu cometer erros na validação de bilhetes com o telemóvel é baixo.	
\item O risco de não receber o bilhete ou receber com atraso é baixo.
\end{enumerate}

~\\\textbf{Que vantagens associa à compra e validação de bilhetes de transporte através do telemóvel?}
 
~\\\textbf{Que desvantagens associa à compra e validação de bilhetes de transporte através do telemóvel?}
 
~\\\textbf{Género}
\begin{itemize}
\item Feminino
\item Masculino
\end{itemize}

~\\\textbf{Idade}
 
~\\\textbf{Nome}