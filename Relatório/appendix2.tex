\chapter{Especificação da API} \label{api}

\section{Descrição Detalhada}

\subsection{API Comum}

\subsubsection{GetAllLinesPathsByStop}

Este serviço permite obter a listagem de linhas e respetivos trajetos que passam numa determinada paragem. Uma linha pode ter um ou mais trajetos. É utilizado quando o utilizador escolhe a paragem de entrada.
\newline
~\\Parâmetros:
\begin{itemize}
\item stop - código SMS da paragem.
\end{itemize}

~\\Resultado (Lista de elementos):
\begin{itemize}
\item GoPathCode - Código do trajeto;
\item GoTerminal - Nome do destino;
\item LineCode - Código da linha;
\item LineName - Nome da linha (número e denominação).
\end{itemize}

\subsubsection{GetAllNearStops}

Este serviço permite obter a listagem de paragens encontradas num determinado raio centrado nas coordenadas especificadas. É utilizado para obter as paragens mais próximas da localização atual do utilizador, quando este inicia o processo de validação.
\newline
~\\Parâmetros:
\begin{itemize}
\item latitude - latitude do centro;
\item longitude - longitude do centro;
\item radius - raio de inclusão.
\end{itemize}

~\\Resultado (Lista de elementos):
\begin{itemize}
\item code - código SMS da paragem;
\item coordX - latitude da paragem;
\item coordY - longitude da paragem;
\item name - Nome da paragem;
\item provider - Operador da paragem.
\end{itemize}

\subsubsection{LoadStopsByWord}

Este serviço permite obter a listagem de paragens que contenham determinada palavra no nome ou no código SMS. É utilizado quando o utilizador seleciona manualmente a paragem de entrada.
\newline
~\\Parâmetros:
\begin{itemize}
\item word - palavra a pesquisar.
\end{itemize}

~\\Resultado (Lista de elementos):
\begin{itemize}
\item code - código SMS da paragem;
\item coordX - latitude da paragem;
\item coordY - longitude da paragem;
\item name - Nome da paragem;
\item provider - Operador da paragem.
\end{itemize}

\subsubsection{GetAllProvidersName}

Este serviço permite obter a listagem completa de operadores de transportes públicos que operam no sistema Andante. É utilizado quando o utilizador seleciona manualmente a paragem de entrada.
\newline
~\\Resultado (Lista de Strings):
\begin{itemize}
\item nomes dos operadores.
\end{itemize}

\subsubsection{GetAllLinesByProvider}

Este serviço permite obter a listagem completa de linhas que um determinado operador tem. É utilizado quando o utilizador seleciona manualmente a paragem de entrada.
\newline
~\\Parâmetros:
\begin{itemize}
\item provider - Nome do operador.
\end{itemize}

~\\Resultado (Lista de Strings):
\begin{itemize}
\item códigos das linhas.
\end{itemize}

\subsubsection{GetAllStopsByLine}

Este serviço permite obter a listagem completa de paragens que constituem determinada linha (em todos os trajetos da linha). É utilizado quando o utilizador seleciona manualmente a paragem de entrada.
\newline
~\\Parâmetros:
\begin{itemize}
\item line - código da linha.
\end{itemize}

~\\Resultado (Lista de elementos):
\begin{itemize}
\item code - código SMS da paragem;
\item coordX - latitude da paragem;
\item coordY - longitude da paragem;
\item name - Nome da paragem;
\item provider - Operador da paragem.
\end{itemize}

\subsection{API Específica}

\subsubsection{login}

Este serviço permite verificar os dados de login introduzidos e confirmar/rejeitar a autenticação do utilizador. É utilizado quando o utilizador se autentica na aplicação quando esta não tem nenhuma conta ativa.
\newline
~\\Parâmetros:
\begin{itemize}
\item email - email do utilizador;
\item password - password do utilizador.
\end{itemize}

~\\Resultado:
\begin{itemize}
\item Code - código da operação:
\subitem 2100 - Operação realizada com sucesso;
\subitem 2101 - Email inexistente;
\subitem 2102 - Password e email não coincidem.
\end{itemize}

\subsubsection{newUser}

Este serviço permite criar um novo utilizador, verificando a existência do email, o elemento único identificativo. É utilizado quando o utilizador se regista na aplicação.
\newline
~\\Parâmetros:
\begin{itemize}
\item email - email do utilizador;
\item password - password do utilizador;
\item pin - pin do utilizador;
\item name - nome do utilizador;
\item address - morada do utilizador;
\item mobile - número de telemóvel do utilizador;
\item NIF - Número de Informação Fiscal do utilizador;
\item birthDate - data de nascimento do utilizador;
\item maxAmount - valor máximo para operações sem necessidade de introduzir pin.
\end{itemize}

~\\Resultado:
\begin{itemize}
\item Code - código da operação:
\subitem 3000 - Operação realizada com sucesso;
\subitem 3001 - Email já existente.
\end{itemize}

\subsubsection{addTicket}

Este serviço permite adicionar títulos ocasionais à conta de um determinado utilizador. É utilizado quando o utilizador compra títulos ocasionais.
\newline
~\\Parâmetros:
\begin{itemize}
\item email - email do utilizador;
\item type - tipo do título;
\item name - tipologia do título;
\item amount - quantidade a adicionar.
\end{itemize}

~\\Resultado:
\begin{itemize}
\item Code - código da operação:
\subitem 4000 - Operação realizada com sucesso;
\subitem 4001 - Email inexistente;
\subitem 4002 - Saldo insuficiente.
\end{itemize}

\subsubsection{addSignature}

Este serviço permite adicionar uma assinatura mensal à conta de um determinado utilizador. É utilizado quando o utilizador compra assinatura mensal.
\newline
~\\Parâmetros:
\begin{itemize}
\item email - email do utilizador;
\item zones - lista de zonas a adicionar.
\end{itemize}

~\\Resultado:
\begin{itemize}
\item Code - código da operação:
\subitem 5000 - Operação realizada com sucesso;
\subitem 5001 - Email inexistente;
\subitem 5002 - Saldo insuficiente.
\end{itemize}


\subsubsection{addMoneyUser}

Este serviço permite adicionar saldo à carteira virtual. É utilizado quando a conta é carregada.
\newline
~\\Parâmetros:
\begin{itemize}
\item email - email do utilizador;
\item amountToAdd - quantidade a adicionar.
\end{itemize}

~\\Resultado:
\begin{itemize}
\item Code - código da operação:
\subitem 6000 - Operação realizada com sucesso;
\subitem 6001 - Email inexistente.
\end{itemize}

\subsubsection{editUser}

Este serviço permite editar os dados pessoais do utilizador. É utilizado quando o utilizador altera os seus dados pessoais nas definições da aplicação.
\newline
~\\Parâmetros:
\begin{itemize}
\item email - email do utilizador;
\item password - password do utilizador;
\item pin - pin do utilizador;
\item name - nome do utilizador;
\item address - morada do utilizador;
\item mobile - número de telemóvel do utilizador;
\item NIF - Número de Informação Fiscal do utilizador;
\item birthDate - data de nascimento do utilizador;
\item maxAmount - valor máximo para operações sem necessidade de introduzir pin.
\end{itemize}

~\\Resultado:
\begin{itemize}
\item Code - código da operação:
\subitem 3200 - Operação realizada com sucesso;
\subitem 3201 - Email inexistente.
\end{itemize}

\subsubsection{validate}

Este serviço permite validar um título da carteira de títulos. É utilizado após a seleção da paragem de entrada, da linha e trajeto e do título a validar.
\newline
~\\Parâmetros:
\begin{itemize}
\item email - email do utilizador;
\item ticketType - tipo do título;
\item ticketName - tipologia do título;
\item path - trajeto de entrada;
\item stop - paragem de entrada;
\item firstPath - trajeto inicial;
\item firstStop - paragem inicial;
\item isNewValidation - indicação de nova validação ou transbordo.
\end{itemize}

~\\Resultado:
\begin{itemize}
\item Code - código da operação:
\subitem 4200 - Operação realizada com sucesso;
\subitem 4002 - Título inexistente;
\subitem 4004 - Email inexistente.
\item Validation - dados da validação:
\subitem email - email do utilizador;
\subitem lastAllowedStopName - última paragem permitida;
\subitem originStop - paragem inicial;
\subitem path - trajeto de entrada;
\subitem seq - número sequencial;
\subitem ticketName - tipologia do título (Z2, Z3, etc.);
\subitem ticketType - tipo do título (Ocasional, Andante 24 ou Assinatura);
\subitem when - data da validação.
\end{itemize}

\subsubsection{getPrices}

Este serviço permite obter a listagem de preços das várias modalidades do sistema Andante. É utilizado quando o utilizador procede à compra de títulos.
\newline
~\\Resultado:
\begin{itemize}
\item Code - código da operação:
\subitem 3260 - Operação realizada com sucesso.
\item TicketList - lista de títulos:
\subitem Id - identificador do título;
\subitem Name - tipologia do título (Z2, Z3, etc.);
\subitem Price - preço do título;
\subitem Type - tipo do título (Ocasional, Andante 24 ou Assinatura).
\end{itemize}

\subsubsection{getUser}

Este serviço permite obter a informação pessoal de um determinado utilizador. É utilizado quando o utilizador pretende alterar os seus dados pessoais.
\newline
~\\Parâmetros:
\begin{itemize}
\item email - email do utilizador.
\end{itemize}

~\\Resultado:
\begin{itemize}
\item Code - código da operação:
\subitem 3250 - Operação realizada com sucesso;
\subitem 3251 - Email inexistente.
\item UserInfo - informação do utilizador:
\subitem email - email do utilizador;
\subitem password - password do utilizador;
\subitem pin - pin do utilizador;
\subitem name - nome do utilizador;
\subitem address - morada do utilizador;
\subitem mobile - número de telemóvel do utilizador;
\subitem NIF - Número de Informação Fiscal do utilizador;
\subitem birthDate - data de nascimento do utilizador;
\subitem maxAmount - valor máximo para operações sem necessidade de introduzir pin.
\end{itemize}

\subsubsection{getTickets}

Este serviço permite obter a listagem de títulos disponíveis de um determinado utilizador. É utilizado quando o utilizador pretende proceder a uma validação, ver o saldo atual de títulos ou proceder à compra dos mesmos.
\newline
~\\Parâmetros:
\begin{itemize}
\item email - email do utilizador.
\end{itemize}

~\\Resultado:
\begin{itemize}
\item Code - código da operação:
\subitem 4100 - Operação realizada com sucesso;
\subitem 4102 - Email inexistente.
\item ListTickets - lista de títulos:
\subitem Amount - identificador do título;
\subitem Name - tipologia do título (Z2, Z3, etc.);
\subitem SigMonth - mês da assinatura (no caso de ser assinatura, 0 nos outros);
\subitem SigYear - ano da assinatura (no caso de ser assinatura, 0 nos outros);
\subitem SigZones - zonas da assinatura (no caso de ser assinatura, null nos outros);
\subitem Type - tipo do título (Ocasional, Andante 24 ou Assinatura).
\end{itemize}

\subsubsection{getAccountMovements}

Este serviço permite obter as transações efetuadas por um determinado utilizador. É utilizado quando o utilizador pretende consultar o histórico de operações realizadas na sua conta.
\newline
~\\Parâmetros:
\begin{itemize}
\item email - email do utilizador.
\end{itemize}

~\\Resultado:
\begin{itemize}
\item Code - código da operação:
\subitem 4150 - Operação realizada com sucesso;
\subitem 4152 - Email inexistente.
\item ListAccountMovements - Lista de operações:
\subitem amount - valor da operação;
\subitem reason - descritivo da operação (carregamento, compra, etc.);
\subitem timeStamp - data da operação.
\end{itemize}


\subsubsection{getAccountValidations}

Este serviço permite obter a listagem de validações de um determinado utilizador. É utilizado quando o utilizador pretende consultar o histórico de validações da sua assinatura ou dos títulos ocasionais.
\newline
~\\Parâmetros:
\begin{itemize}
\item email - email do utilizador.
\end{itemize}

~\\Resultado:
\begin{itemize}
\item Code - código da operação:
\subitem 4150 - Operação realizada com sucesso;
\subitem 4152 - Email inexistente.
\item ListAccountValidations - Lista de validações:
\subitem Id - número sequencial;
\subitem PathCode - código do trajeto;
\subitem Signature - indicação de assinatura ou título ocasional;
\subitem StopCode - código da paragem;
\subitem TicketName - tipologia do título (Z2, Z3, etc.);
\subitem TicketType - tipo do título (Ocasional, Andante 24 ou Assinatura);
\subitem TimeStamp - data da validação;
\subitem isNewValidation - indicação de nova validação ou transbordo.
\end{itemize}

\subsubsection{getStopZone}

Este serviço permite saber qual a zona Andante de determinada paragem. É utilizado para saber se é possível utilizar a assinatura na paragem selecionada.
\newline
~\\Parâmetros:
\begin{itemize}
\item pathCode - código do trajeto;
\item stopCode - código da paragem.
\end{itemize}

~\\Resultado:
\begin{itemize}
\item Code - código da operação:
\subitem 1050 - Operação realizada com sucesso;
\subitem 1051 - Operação sem sucesso.
\item Zona da paragem.
\end{itemize}